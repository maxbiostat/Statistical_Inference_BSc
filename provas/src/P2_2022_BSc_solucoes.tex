\documentclass[a4paper,10pt, notitlepage]{report}
\usepackage[utf8]{inputenc}
\usepackage{natbib}
\usepackage{amssymb}
\usepackage{amsmath}
\usepackage{enumitem}
\usepackage{xcolor}
\usepackage{url}
\usepackage{cancel}
\usepackage{mathtools}
\usepackage[portuguese]{babel}
\usepackage{newclude}

%%%%%%%%%%%%%%%%%%%% Notation stuff
\newcommand{\pr}{\operatorname{Pr}} %% probability
\newcommand{\vr}{\operatorname{Var}} %% variance
\newcommand{\rs}{X_1, X_2, \ldots, X_n} %%  random sample
\newcommand{\irs}{X_1, X_2, \ldots} %% infinite random sample
\newcommand{\rsd}{x_1, x_2, \ldots, x_n} %%  random sample, realised
\newcommand{\bX}{\boldsymbol{X}} %%  random sample, contracted form (bold)
\newcommand{\bx}{\boldsymbol{x}} %%  random sample, realised, contracted form (bold)
\newcommand{\bT}{\boldsymbol{T}} %%  Statistic, vector form (bold)
\newcommand{\bt}{\boldsymbol{t}} %%  Statistic, realised, vector form (bold)
\newcommand{\emv}{\hat{\theta}}
\DeclarePairedDelimiter\ceil{\lceil}{\rceil}
\DeclarePairedDelimiter\floor{\lfloor}{\rfloor}
\DeclareMathOperator*{\argmax}{arg\,max}
\DeclareMathOperator*{\argmin}{arg\,min}
%%%%
\newif\ifanswers
\answerstrue % comment out to hide answers

% Title Page
\title{Segunda avaliação (A2)}
\author{Disciplina: Inferência Estatística \\ Instrutor: Luiz Max Carvalho \\ Monitores: Jairon Nóia \& Tiago Silva}
\date{26 de Novembro de 2022}

\begin{document}
\maketitle

\begin{center}
\fbox{\fbox{\parbox{1.0\textwidth}{\textsf{
    \begin{itemize}
        \item O tempo para realização da prova é de 3 horas;
        \item Leia a prova toda com calma antes de começar a responder;
        \item Responda todas as questões sucintamente;
        \item Marque a resposta final claramente com um quadrado, círculo ou figura geométrica de sua preferência;
        \item A prova vale 80 pontos. A pontuação restante é contada como bônus;
        \item Apenas tente resolver a questão bônus quando tiver resolvido todo o resto;
        \item Você tem direito a trazer \textbf{\underline{uma} folha de ``cola''} tamanho A4 frente e verso, que deverá ser entregue junto com as respostas da prova.
    \end{itemize}}
}}}
\end{center}

\newpage

\section*{1. O estatístico e o poeta.}

        \begin{center}\textit{
        Eu te vejo sumir por aí\\
        Te avisei que a cidade era um vão\\
        Dá tua mão, olha pra mim\\
        Não faz assim, não vai lá, não\\
        Os letreiros a te colorir\\
        Embaraçam a minha visão\\
        Eu te vi suspirar de aflição\\
        E sair da sessão frouxa de rir\\
        Já te vejo brincando gostando de ser\\
        Tua sombra a se multiplicar\\
        Nos teus olhos também posso ver\\
        As vitrines te vendo passar\\
        Na galeria, cada clarão\\
        É como um dia depois de outro dia\\
        Abrindo um salão\\
        Passas em exposição\\
        Passas sem ver teu vigia\\
        Catando a poesia\\
        Que entornas no chão\\
        }
        \end{center}
        \textit{As Vitrines (Almanaque, 1981)} de Chico Buarque (1944-).\\            
        
O eu-lírico da canção, que vamos chamar aqui de Ivo, pensa em seu amado, Adão.
Adão é poeta, e tem a estranha mania de deixar cair seus poemas ao passear pelo shopping.
Ivo, muito solícito e perdidamente apaixonado, corre atrás do companheiro catando os papéis que
o desastrado deixa cair.
Sendo estatístico, Ivo sabe que pode modelar o tempo entre a queda dos poemas como uma variável aleatória exponencial com taxa $\theta$.
Ivo quer saber se será capaz de acompanhar Adão na sua jornada sem perder nenhum poema.
Para isso, julga que se $\theta \leq \theta_0$, ele será capaz de catar toda a poesia deixada por Adão antes de ser carregada pelo vento.

Suponha que Ivo observa o processo de queda de $n$ poemas e anota o tempo entre cada queda, formando a amostra $Y_1, Y_2, \ldots, Y_n$.
Ivo considera a estatística de teste $S = \sum_{i=1}^n Y_i$ e constrói o teste $\delta_c$ de modo que, se $S \geq c$, ele rejeita a hipótese $H_0: \theta \leq \theta_0$.

\begin{enumerate}[label=\alph*)]
 \item (10 pontos) Encontre a função poder do teste de Ivo.
 \item (10 pontos) Mostre que a função poder do item anterior é~\textbf{não-decrescente} em $\theta$;
 
 \textbf{Dica:} Se $X$ tem distribuição Gama com parâmetros $k \in \mathbb{N}$ e $\theta$, então 
 \begin{equation*}
     P_\theta \left(X \leq x \right) = e^{-x/\theta}\sum_{j = k}^\infty \frac{1}{j!}\left(\frac{x}{\theta}\right)^j.
 \end{equation*}
 \item (10 pontos) Encontre uma expressão para o tamanho $\alpha_0$ do teste $\delta_c$;
 \item (10 pontos) O teste em questão é não-viesado? Justifique;
\end{enumerate}
\ifanswers
\include*{A2_2022_sol1}
\fi

\section*{2. PO-KÉ-MON!} 

Suponha que a Liga Internacional de Pokemon (LIP) tenha um sistema de \textit{pokescores} que podem assumir qualquer valor real. 
Quanto maior o \textit{pokescore} de uma jogadora, mais alto no ranking mundial ela está.
A liga se organiza em times de $n$ jogadores.

Para entrar na liga, um time precisa ter um \textit{pokescore} médio superior a $\theta_0$, isto é, a média dos pokescores de seus jogadores precisa ser maior que $\theta_0$.
Suponha que os \textit{pokescores} dentro de um time são distribuídos de acordo com uma distribuição Normal com média $\theta$ e variância $\sigma^2$, conhecida.
Queremos desenvolver um método para incluir times num torneio automaticamente, baseado nos \textit{pokescores} dos seus integrantes.

\begin{enumerate}[label=\alph*)] 
 \item (5 pontos) Encontre uma quantidade pivotal para $\theta$;
 \item (5 pontos) Utilizando a quantidade do item anterior, construa um intervalo de confiança de $95\%$ para $\theta$;
 \item (10 pontos) A partir do intervalo encontrado, é possível testar $H_0: \theta \leq \theta_0$? Como?
 \item (10 pontos) Se $\sigma^2$ fosse desconhecida, como você modificaria o teste do item anterior?
 \item (5 pontos) Se aplicarmos os testes em (c) e (d) para selecionar times automaticamente, seremos injustos com alguns times, isto é, vamos deixar de incluir times que de fato se encaixam na condição de seleção.
 Com que probabilidade isso acontece?
 \item (5 pontos) Se quisermos diminuir a probabilidade do item anterior, o que podemos fazer? Que consequências isso tem?
\end{enumerate} 
\ifanswers
\include*{A2_2022_sol2}
\fi

\section*{3. Run, Joey, run!\footnote{Linear regression is a war horse of Statistics. The horse in `War Horse' (2011) is named Joey.}}

O modelo linear (de regressão) é um dos cavalos de batalha da Estatística, sendo aplicado em problemas de Finanças, Medicina e Engenharia.
Vamos agora estudar como utilizar as propriedades deste modelo para desenhar experimentos com garantias matemáticas de desempenho e obter estimadores de quantidades de interesse.

\begin{enumerate}[label=\alph*)]
 \item (10 pontos) Uma prática comum em regressão é a de \textbf{centrar} a variável independente (covariável), isto é subtrair a média; isto facilita a interpretação do intercepto e também simplifica alguns cálculos importantes.
 Mostre que no caso com a covariável centrada, $\hat{\beta_0}$ e $\hat{\beta_1}$ são independentes;
 \item (10 pontos) Mais uma vez considerando o caso centrado, mostre 
 como obter o número de observações $n$ que faz com que a variância do estimador de máxima verossimilhança do intercepto seja menor que $v > 0$;
 \item (10 pontos) Mostre como obter um estimador não-viesado da quantidade $\theta = a\beta_0 + b\beta_1 + c$, com $a, b, c \neq 0$, e encontre o seu erro quadrático médio.
 \item (10 pontos)  Quando $x_{\text{pred}} = \bar{x}$, mostre como obter  o número de observações $n$ necessário para que o intervalo de predição de $100(1-\alpha_0)\%$ para a variável-resposta ($Y$) tenha largura menor ou igual a $l>0$ com probabilidade pelo menos $\gamma$.
 
 \textit{Dicas}:(i) A expressão dependerá~\textit{também} da variância dos resíduos, $\sigma^2$ e (ii) Você não precisa calcular $n$, apenas mostrar o procedimento para obtê-lo.
 \end{enumerate}


\ifanswers
\include*{A2_2022_sol3}
\fi

% \bibliographystyle{apalike}
% \bibliography{refs}

\end{document}          
