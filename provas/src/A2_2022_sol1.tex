\textcolor{red}{\textbf{Conceitos trabalhados}: função poder; tamanho.}
\textcolor{purple}{\textbf{Nível de dificuldade}: fácil.}\\
\textcolor{blue}{
\textbf{Resolução:}
Para responder a), vamos lembrar que a função poder $\pi(\theta \mid \delta_c) = P_\theta\left(\textrm{Rejeitar}\: H_0\right)$.
Sendo assim, temos
\begin{align*}
    \pi(\theta \mid \delta_c) &= P_\theta\left(S \geq c\right),\\
    &= 1 - P_\theta(S < c),\\
    &= 1 - F_S\left(c; n, \theta \right),
\end{align*}
onde $F_S\left(x; a, b\right)$ é a f.d.a. de uma distribuição Gama com forma $a$ e taxa $b$ avaliada em $x \in \mathbb{R}$. 
Agora precisamos mostrar que $\pi(\theta \mid \delta_c)$ é não descrescente em $\theta$ de modo a responder b).
Usando a dica, sabemos que 
\begin{equation*}
    \pi(\theta \mid \delta_c) = 1 - e^{-c/\theta}\sum_{j = k}^\infty \frac{1}{j!}\left(\frac{c}{\theta}\right)^j,
\end{equation*}
de modo que $\frac{\partial}{\partial \theta}\pi(\theta \mid \delta_c) \geq 0$.
Outro bom argumento é esboçar o gráfico da função poder e mostrar que ela não pode decrescer.
O tamanho de $\delta_c$ é dado por
\begin{equation*}
    \alpha_0 := \sup_{\theta \in \Theta_0} \pi(\theta \mid \delta_c).
\end{equation*}
Como a função poder é não descrescente, temos que $\alpha_0 = \pi(\theta_0 \mid \delta_c)$, respondendo c).
Em d), temos que o teste de fato é não-viesado, pois a função poder é não descrescente em $\theta$, de modo que para todo par $\theta \in \Theta \setminus \Theta_0$ e $\theta^\prime \in \Theta_0$ temos que $\pi(\theta^\prime \mid \theta) \leq \pi(\theta \mid \theta)$.
$\blacksquare$\\
\textbf{Comentário:} Esta é uma questão parecida com a Q1 da A2 de 2020, mas neste caso Ivo mede os tempos entre as quedas dos poemas. Uma questão simples e conceitual para esquentar os músculos.
}