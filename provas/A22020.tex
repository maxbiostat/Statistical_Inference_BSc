\documentclass[a4paper,10pt, notitlepage]{report}
\usepackage[utf8]{inputenc}
\usepackage{natbib}
\usepackage{amssymb}
\usepackage{amsmath}
\usepackage{enumitem}
\usepackage{xcolor}
\usepackage{cancel}
\usepackage{mathtools}
\usepackage{float}
\usepackage[portuguese]{babel}

%%%%%%%%%%%%%%%%%%%% Notation stuff
\newcommand{\pr}{\operatorname{Pr}} %% probability
\newcommand{\vr}{\operatorname{Var}} %% variance
\newcommand{\rs}{X_1, X_2, \ldots, X_n} %%  random sample
\newcommand{\irs}{X_1, X_2, \ldots} %% infinite random sample
\newcommand{\rsd}{x_1, x_2, \ldots, x_n} %%  random sample, realised
\newcommand{\bX}{\boldsymbol{X}} %%  random sample, contracted form (bold)
\newcommand{\bx}{\boldsymbol{x}} %%  random sample, realised, contracted form (bold)
\newcommand{\bT}{\boldsymbol{T}} %%  Statistic, vector form (bold)
\newcommand{\bt}{\boldsymbol{t}} %%  Statistic, realised, vector form (bold)
\newcommand{\emv}{\hat{\theta}}
\DeclarePairedDelimiter\ceil{\lceil}{\rceil}
\DeclarePairedDelimiter\floor{\lfloor}{\rfloor}

% Title Page
\title{Segunda avaliação (A2)}
\author{Disciplina: Inferência Estatística \\ Instrutor: Professor Carvalho}
\date{03 de Dezembro de 2020}

\begin{document}
\maketitle

% \textbf{Data de Entrega: 19 de Agosto de 2020.}

\begin{center}
\fbox{\fbox{\parbox{1.0\textwidth}{\textsf{
    \begin{itemize}
        \item Por favor, entregue um único arquivo PDF;
        \item O tempo para realização da prova é de 3 horas, mais vinte minutos para upload do documento para o e-class;
        \item Leia a prova toda com calma antes de começar a responder;
        \item Responda todas as questões sucintamente;
        \item Marque a resposta final claramente com um quadrado, círculo ou figura geométrica de sua preferência;
        \item A prova vale 80 pontos. A pontuação restante é contada como bônus;
        \item Apenas tente resolver a questão bônus quando tiver resolvido todo o resto;
        \item Lembre-se de consultar o catálogo de fórmulas no fim deste documento.
    \end{itemize}}
}}}
\end{center}
\newpage
\section*{Dicas}
\begin{itemize}
 \item Em uma regressão linear simples, temos:
  \begin{align*}
  \hat{\beta_0} &\sim \operatorname{Normal}\left(\beta_0, \sigma^2 \left( \frac{1}{n} + \frac{\bar{x}^2}{s_x^2} \right) \right),\\
  \hat{\beta_1}  &\sim \operatorname{Normal}\left(\beta_1, \frac{\sigma^2}{s_x^2}\right),\\
  &\operatorname{Cov}\left(\hat{\beta_0}, \hat{\beta_1} \right)  = -\frac{\bar{x}\sigma^2}{s_x^2},
 \end{align*}
 onde $s_x = \sqrt{\sum_{i=1}^n (x_i-\bar{x})^2}$ e $\hat{\beta_0}$ e $\hat{\beta_1}$ são os estimadores de máxima verossimilhança dos coeficientes.
 \item Um processo de Poisson com taxa $\lambda$ por unidade de tempo é um processo estocástico que satisfaz:
 \begin{itemize}
  \item O número de chegadas em um intervalo de tempo $\Delta_t$ tem distribuição Poisson com média $\lambda\Delta_t$.
  \item Os números de chegadas em qualquer coleção de intervalos disjuntos são independentes.
 \end{itemize}
 \item Se $X$ tem distribuição Poisson com média $\lambda$, então
 \[\pr\left(X \leq x\right) = Q(\floor*{x + 1}, \lambda), \]
 onde $$Q(x, s) = \frac{\Gamma(x, s)}{\Gamma(x)}$$ é a função Gama regularizada superior e $\floor*{y}$ é maior inteiro menor ou igual a $y$ -- também chamado de~\textit{floor}.
 Ademais, temos
\[ \frac{\partial}{\partial s} Q(x, s) = -\frac{e^{-s}s^{x-1}}{\Gamma(x)},\]
onde $\Gamma(x) = (x-1)!$ é a função Gamma.
 \end{itemize}
 
\newpage

\section*{1. Catando poesia.}

        \begin{center}\textit{
        Eu te vejo sumir por aí\\
        Te avisei que a cidade era um vão\\
        Dá tua mão, olha pra mim\\
        Não faz assim, não vai lá, não\\
        Os letreiros a te colorir\\
        Embaraçam a minha visão\\
        Eu te vi suspirar de aflição\\
        E sair da sessão frouxa de rir\\
        Já te vejo brincando gostando de ser\\
        Tua sombra a se multiplicar\\
        Nos teus olhos também posso ver\\
        As vitrines te vendo passar\\
        Na galeria, cada clarão\\
        É como um dia depois de outro dia\\
        Abrindo um salão\\
        Passas em exposição\\
        Passas sem ver teu vigia\\
        Catando a poesia\\
        Que entornas no chão\\
        }
        \end{center}
        \textit{As Vitrines (Almanaque, 1981)} de Chico Buarque (1944-).\\            
        
O eu-lírico da canção, que vamos chamar aqui de Ivo, pensa em seu amado, Adão.
Adão é poeta, e tem a estranha mania de deixar cair seus poemas ao passear pelo shopping.
Ivo, muito solícito e perdidamente apaixonado, corre atrás do companheiro catando os papéis que
o desastrado deixa cair.
Sendo estatístico, Ivo sabe que pode modelar o processo de queda dos poemas como um processo de Poisson com média $\theta$.
Ivo quer saber se será capaz de acompanhar Adão na sua jornada sem perder nenhum poema.
Para isso, julga que se $\theta \leq \theta_0$, ele será capaz de catar toda a poesia deixada por Adão antes de ser carregada pelo vento.

Suponha que Ivo observa o processo de queda dos poemas em $n$ intervalos de exatamente $t$ unidades de tempo e toma nota dos números $Y_1, Y_2, \ldots, Y_n$ de poemas caídos em cada intervalo.
Ivo considera a estatística de teste $S = \sum_{i=1}^n Y_i$ e constrói o teste $\delta_c$ de modo que, se $S \geq c$, ele rejeita a hipótese $H_0: \theta \leq \theta_0$.

\begin{enumerate}[label=\alph*)]
 \item (10 pontos) Encontre a função poder do teste de Ivo.
 \item (10 pontos) Mostre que a função poder do item anterior é~\textbf{não-decrescente} em $\theta$;
 \item (2,5 pontos) Encontre uma expressão para o tamanho $\alpha_0$ do teste $\delta_c$;
 \item (2,5 pontos) O teste em questão é não-viesado? Justifique;
 \item (5 pontos) Discuta se é possível atingir qualquer tamanho para $\delta_c$ e o que fazer se queremos um tamanho de, por exemplo, $\alpha_0 = 0.01$.
\end{enumerate}

\section*{2.Temos que pegar!}

Além de apaixonados um pelo outro, Joelinton e Valcicléia também amam Pokémon.
Os dois jogam competitivamente na Liga Brasileira de Pokémon (LBP).
Há, contudo, um pequeno incoveniente:  Joelinton é~\textit{Team Magma} enquanto Valcicléia é~\textit{Team Aqua}.
Durante uma conversa acalorada, Joelinton afirma que o~\textit{Team Magma} é melhor, em termos de \textit{pokescores} médios, que o~\textit{Team Aqua}.
Valcicléia propõe consultar o site da LBP para obter dados sobre o assunto.
Ao consultar o site, eles obtem $m$ valores de \textit{pokescores} de integrantes do~\textit{Team Magma} e $n$ valores de integrantes do~\textit{Team Aqua}.

Suponha que modelamos os \textit{pokescores} de cada jogador(a) em cada time como variáveis aleatórias normais com médias $\mu_{M}$ e $\mu_{A}$ e variâncias $\sigma_{M}^2$ e $\sigma_{A}^2$, respectivamente.
Nos itens a seguir,~\textbf{enuncie claramente} qual é a hipótese nula -- e a hipótese alternativa -- em cada caso, qual é a estatística de teste e qual o procedimento de teste.

\begin{enumerate}[label=\alph*)]
 \item (2,5 pontos) A partir do desenho experimental descrito, encontre quantidades pivotais para $\mu_M$ e $\mu_A$, supondo $\sigma_M^2$ e $\sigma_A^2$~\textbf{desconhecidas}. 
 Justifique;
 \item (2,5 pontos) Utilize as quantidades do item anterior para construir intervalos de confiança exatos de 99\% para $\mu_A$ e $\mu_M$;
  \item (5 pontos) Suponha que, no calor do momento, Valcicléia afirme que o~\textit{Team Magma} é tão ruim que não tem pokescore médio suficiente nem para competir na Liga Regional de Pokemon (LRP).
  Sabendo que o pokescore médio necessário para admissão na LRP é $\mu_0$, mostre a Joelinton como utilizar o intervalo de confiança obtido no item anterior para testar a hipótese levantada por sua amada;
 \item (10 pontos) Nossa dupla dinâmica está interessada em comparar as médias supondo que as variâncias são iguais.
 Proponha um teste de tamanho $\alpha_0$ para avaliar a premissa de homogeneidade (variâncias iguais);
 \item (10 pontos) Suponha que o teste do item anterior falhou em rejeitar $H_0$.
 Proponha um teste de tamanho $\alpha_0$ para testar a hipótese inicial de Joelinton;
\end{enumerate}

\newpage
\section*{3. Regressão linear: o melhor modelo ruim que você já viu.}

Considere a figura a seguir:
 \begin{figure}[H]
  \begin{center}
    \includegraphics[scale=.65]{../slides/figures/anscombe_mod.pdf}
  \end{center}
   \end{figure}

   
Em todos os painéis, $\bar{x} = 9$, $\bar{y} = 7,5$, $s_x^2 = 110$ e $\operatorname{Cor}(X, Y) = 0,816$.
Isto é, todas as estatísticas sumárias relevantes atingem os mesmos valores.
Disto resulta que $\hat{\beta_0} = 3$ e $\hat{\beta_1}=0,5$ para todos os painéis.

\begin{enumerate}[label=\alph*)]
 \item (15 pontos) Comente sobre quais premissas básicas -- ou nenhuma -- da regressão linear aparentam estar sendo violadas em cada painel.
Justifique.
 \item (5 pontos) Os estimadores de máxima verossimilhança para os coeficientes no modelo
 \[ E[Y_i] = \beta_0 + \beta_1 X_i \]
 são 
 \begin{align*}
  \hat{\beta_0} &= \bar{y} - \hat{\beta_1}\bar{x},\\
  \hat{\beta_1} &= \frac{\sum_{i=1}^n (Y_i-\bar{y})(X_i-\bar{x})}{\sum_{i=1}^n \left(X_i - \bar{x}\right)^2}.
 \end{align*}
 Tais estimadores são viesados?
 Justifique.
 
 \textit{Dica:} Pode ser conveniente escrever
 $$\hat{\beta_1} = \frac{\sum_{i=1}^n \left(X_i-\bar{x}\right)Y_i}{s_x^2}.$$
 \end{enumerate}

\section*{Questão Bônus: uma transformação útil.} 

Muitas vezes na aplicação de modelos de regressão é conveniente aplicar uma transformação à(s) variável(is) independente(s) de modo a facilitar a computação e/ou a interpretação das estimativas.

\begin{enumerate}[label=\alph*)]
 \item (10 pontos)  Considere uma regressão linear simples.
 Encontre  uma transformação $X^\prime = f(X)$ da variável independente de modo que $\hat{\beta_0^\prime}$ e $\hat{\beta_1^\prime}$ sejam independentes. 
 \item (5 pontos) Encontre o valor de $\hat{\beta_0^\prime}$ e $\hat{\beta_1^\prime}$ sob a transformação do item anterior.
 \item (5 pontos) Como essa transformação muda a interpretação dos coeficientes estimados? 
\end{enumerate}

\section*{Fórmulas úteis}
\textbf{Como usar este catálogo:} as fórmulas dadas aqui estão propositalmente privadas do seu contexto.
O objetivo desta coleção é ajudar você a lembrar das expressões.
Entretanto, saber quais expressões são utilizadas em que contexto é sua tarefa.
\begin{itemize}
 \item $ \bar{X}_n = \frac{1}{n} \sum_{i=1}^n X_i$;
 \item $\hat{\sigma}^\prime = \sqrt{\frac{1}{n-1}\sum_{i=1}^n \left(X_i - \bar{X}_n\right)^2}$;
 \item $S_X^2 = \sum_{i=1}^m (X_i-\bar{X}_m)^2$;
 \item $S_Y^2 = \sum_{j=1}^n (Y_j-\bar{Y}_n)^2$;
 \item $U = \frac{\sqrt{m + n - 2}(\bar{X}_m - \bar{Y}_n)}{\sqrt{\left(\frac{1}{m} + \frac{1}{n}\right) (S_X^2 + S_Y^2)}}$;
 \item $ V = \frac{S_X^2/(m-1)}{S_Y^2/(n-1)}$;
 \item $\bar{x} = (1/n)\sum_{i=1}^n X_i$;
 \item $\bar{y} = (1/n)\sum_{i=1}^n Y_i$.
\end{itemize}


% \bibliographystyle{apalike}
% \bibliography{refs}

\end{document}          
