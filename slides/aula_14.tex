\section{Testes de hipóteses}
\begin{frame}{Testes de hipóteses}
 \begin{itemize}
  \item Hipótese nula e alternativa;
  \item Hipóteses simples e compostas;
  \item Região crítica e estatística teste;
  \item Função poder;
  \item Tipos de erro (I e II);
  \item P-valor;
   \end{itemize}
\end{frame}

\begin{frame}{Hipótese nula e alternativa}
 No teste de hipóteses estatísticas, identificamos partições do espaço de parâmetros que codificam as hipóteses de interesse.
 \begin{defn}[Hipótese nula e hipótese alternativa]
  Considere o espaço de parâmetros $\Omega$ e defina $\Omega_0, \Omega_1 \subset \Omega$ de modo que $\Omega_0 \cup \Omega_1 = \Omega$ e $\Omega_0 \cap \Omega_1 = \emptyset$.
  Definimos
  \begin{align*}
   H_0 &:= \theta \in \Omega_0,\\
   H_1 &:= \theta \in \Omega_1.
  \end{align*}
Dizemos que $H_0$ é a~\textbf{hipótese nula} e $H_1$ é a~\textbf{hipótese alternativa}.

Se $\theta \in \Omega_1$, dizemos que~\textit{rejeitamos} a hipótese nula.
Por outro lado, se $\theta \in \Omega_0$ dizemos que~\textit{não rejeitamos} ou~\textit{falhamos em rejeitar} $H_0$.
 \end{defn}
\end{frame}

\begin{frame}{Exemplo}

Suponha que Palmirinha recebeu uma carta da Associação Nacional da Pamonha Gourmet (ANPG), dizendo que a pamonha deve ter, no mínimo, 7 mg/L de concentração de amido.
Supondo que a concentração de amido tenha distribuição Normal com parâmetros $\mu$ (desconhecido) e $\sigma^2$ (conhecido), Palmirinha rabisca num papel:
  \begin{align*}
   H_0 &:= \mu \in [7, \infty),\\
   H_1 &:= \mu \in (0, 7).
  \end{align*}
\end{frame}

\begin{frame}{Hipóteses simples e compostas}
 Dependendo do tipo de partição do espaço de parâmetros, as hipóteses recebem classificações diferentes.
 \begin{defn}[Hipótese simples e hipótese compostas]
  Dizemos que uma hipótese $H_i$, é~\textbf{simples}, se $\Omega_i = \{ \theta_i \}$, isto é, se a partição correspondente é um ponto. 
  Uma hipótese é dita~\textbf{composta} se não é simples.
 \end{defn}
 
 \begin{exemplo}[Hipótese simples sobre a média]
  Suponha que estamos estudando o efeito de uma droga na redução da pressão arterial.
  Modelamos esta redução como uma variável aleatória $X$ com esperança $E[X] =: \theta$.
  É costumaz testar a hipótese $H_0 : \theta = 0$, que chamamos, especificamente nesse caso, de ``hipótese de efeito nulo''.
 \end{exemplo}
\end{frame}

\begin{frame}{Hipótese unilateral e hipótese bilateral}
 Em analogia com os intervalos de confiança, também podemos entender as hipóteses como sendo unilaterais ou bilaterais.
 \begin{defn}[Hipótese unilateral e hipótese bilateral]
  Uma hipótese da forma $H_0 : \theta \leq \theta_0$ ou $H_0 : \theta \geq \theta_0$ é dita~\textit{unilateral} (``\textit{one-sided}''), enquanto hipóteses da forma $H_0 : \theta \neq \theta_0$ são ditas bilaterais (``\textit{two-sided}'').
 \end{defn}
 
 \begin{obs}[Hipóteses bilaterais como consequência de $H_0$ simples]
 Se $H_0$ é simples, a hipótese alternativa $H_1$ será, em geral, bilateral.  
 \end{obs}
\end{frame}

\begin{frame}{Região crítica: exemplo motivador}
 \begin{exemplo}[Teste para a média de uma Normal com variância conhecida]
  Suponha que $\bX = \{ \rs \}$ é uma amostra aleatória de uma Normal com média $\mu$ e variância $\sigma^2$ conhecida.
  Queremos testar a hipótese
  \begin{align*}
   H_0 &:= \mu = \mu_0\\
   H_1 &:= \mu \neq \mu_0.
  \end{align*}
  Intuitivamente, queremos rejeitar $H_0$ se $\Sm$ está longe de $\mu_0$.
  Para isso definimos
  \[ S_0 := \left\{ \bx: -c \leq \Sm - \mu_0 \leq c \right\}, \]
  de modo que $S_1 = S_0^C$. 
  Então, seguimos o procedimento:
      \begin{align*}
   \bX \in S_1 &\implies \text{rejeitar}\: H_0,\\
   \bX \in S_0 &\implies \text{não\:rejeitar}\: H_0.\\
  \end{align*}
 \end{exemplo}
\end{frame}

\begin{frame}{Região crítica e região de rejeição}
 Uma maneira mais simples de expressar o procedimento acima é definir $T := \left|\Sm - \mu_0\right|$ e rejeitar $H_0$ se $T \geq c$.
 \begin{defn}[Região crítica]
  O conjunto 
   \[ S_1 := \left\{ \bx:  \left|\Sm - \mu_0\right| \geq c \right\}, \]
   é chamado de~\textbf{região crítica} do teste.
 \end{defn}
 
Analogamente, considere a estatística $T = r(\bX)$ e tome $R \subseteq \mathbb{R}$. 
Então podemos definir
\begin{defn}[Região de rejeição]
Se $R \subseteq \mathbb{R}$ é tal que dizemos que ``rejeitamos $H_0$ se $T \in R$'', então $R$ é chamada uma~\textbf{região de rejeição} para a estatística $T$ e o teste associado.
\end{defn}
\end{frame}

\begin{frame}{Dividindo o espaço amostral e o espaço de parâmetros}
 Começamos com uma observação:
 \begin{obs}[Correspondência entre região crítica e região de rejeição]
 Podemos relacionar os conceitos de região crítica e região de rejeição notando queremos
   \[ S_1 := \left\{ \bx:  r(\bx) \in R \right\}. \]
\end{obs}

\begin{ideia}[Dividindo o espaço amostral e o espaço de parâmetros]
 Suponha que temos um modelo estatístico dado pela distribuição $f(x\mid\theta)$, com $x \in \mathcal{X}$ e $\theta \in \Omega$.
 Desta forma, uma amostra aleatória $\bX = \{ \rs\}$ mora em $\mathcal{X}^n$.
 Para formular uma hipótese estatística, estabelecemos uma partição do espaço de parâmetros $\Omega$ em $\Omega_0$ e $\Omega_1$ disjuntos.
 Isto, por sua vez, induz uma partição $S_0, S_1 \in \mathcal{X}^n$.
 Estes objetos, embora, relacionados,~\textbf{não são a mesma coisa}.
 Por exemplo, nós observamos se $\bX \in S_0$ ou $\bX \in S_1$, mas raramente ``observamos'' se $\theta \in \Omega_0$ ou $\theta \in \Omega_1$.
\end{ideia}
\end{frame}

% \begin{frame}{Função poder}
%  
% \end{frame}
% 
% \begin{frame}{Tipos de Erro}
%  
% \end{frame}

% \begin{frame}{O que aprendemos?}
% \begin{itemize}
% 
%   \item[\faLightbulbO] ;    
%   
%    ``'';
%     
%   \end{itemize}
%  \end{frame}

\begin{frame}{Leitura recomendada}
\begin{itemize}
 \item[\faBook] De Groot seção 9.1;
%  \item[\faBook] $^\ast$ Casella \& Berger (2002), seção 9.2.
%  \item[\faBook] $^\ast$ Schervish (1995), capítulo 7.
 \item[\faForward] Próxima aula: De Groot, seção 9.1 (razão de verossimilhanças);
 \item {\large\textbf{Exercícios recomendados}}
 \begin{itemize}
  \item[\faBookmark] De Groot.
  \begin{itemize}
   \item Seção 9.1: 3, 8 e 13.
  \end{itemize}   
  \end{itemize}
 \end{itemize} 
\end{frame}
