\section{Intervalos de confiança}
\begin{frame}{Intervalos de confiança}
 \begin{itemize}
  \item Intervalos de confiança;
  \item Caso normal: média;
  \item Intervalos de confiança unilaterais;
  \item Estatística pivotal.  
 \end{itemize}
\end{frame}

\begin{frame}{Intervalo de confiança para a média no caso Normal}
 Lembremos que 
 \begin{equation}
 \label{eq:confidence_interval_normal_mean}
 U = \frac{\sqrt{n}(\Sm-\mu)}{\sqrt{\frac{\Delta^2}{n-1}}} \sim\operatorname{T}(n-1).
\end{equation}
Para $c>0$, podemos computar $\pr(-c < U < c) = \gamma$:
\begin{align*}
 &\pr\left(-c < \frac{\sqrt{n}(\Sm-\mu)}{\sqrt{\frac{\Delta^2}{n-1}}} < c\right) = \gamma,\\
 &\pr\left( \Sm - \frac{c\hat{\sigma}^\prime}{\sqrt{n}} < \mu <  \Sm + \frac{c\hat{\sigma}^\prime}{\sqrt{n}}\right) = \gamma,\\
 &T_{n-1}(c) - T_{n-1}(-c) = 2T_{n-1}(c) - 1 = \gamma.
\end{align*}
Concluímos que $c = F_T^{-1}\left(\frac{1 + \gamma}{2}; n-1\right)$.
\end{frame}

\begin{frame}{Definição de intervalo de confiança}
 O conceito de~\textbf{intervalo de confiança} é fundamental em Estatística e nas aplicações em Ciência.
 \begin{defn}[Intervalo de confiança]
 \label{def:confidence_interval}
  Seja $\bX = \{ \rs \}$ uma amostra aleatória, cada variável aleatória com p.d.f. $f(x\mid \theta)$, e considere uma função real $g(\theta)$.
  Sejam $A(\bX)$ e $B(\bX)$ duas estatísticas de modo que valha
  \begin{equation}
   \label{eq:confidence_interval}
   \pr\left\{A(\bX) < g(\theta) <  B(\bX)\right\} \geq \gamma.
  \end{equation}
Dizemos que $I(\bX) = (A(\bX), B(\bX))$ é um~\textbf{intervalo de confiança} de $100\gamma\%$ para $g(\theta)$.
Se a desigualdade for uma igualdade para todo $\theta \in \Omega$, dizemos que o intervalo é~\textbf{exato}.
 \end{defn}
\end{frame}

\begin{frame}{Revisitando o caso Normal}
No caso do intervalo de confiança para o parâmetro de média, temos 
$$\pr\left\{A(\bX) < g(\mu) <  B(\bX)\right\} \geq \gamma,$$
com $g(\mu) = \mu$  e 
\begin{align*}
 A(\bX) &= \Sm - \frac{c\hat{\sigma}^\prime}{\sqrt{n}} = \Sm - \frac{c\sqrt{\sum_{i=1}^n \left(X_i - \Sm\right)^2}}{\sqrt{n(n-1)}},\\
 B(\bX) &= \Sm + \frac{c\hat{\sigma}^\prime}{\sqrt{n}} = \Sm + \frac{c\sqrt{\sum_{i=1}^n \left(X_i - \Sm\right)^2}}{\sqrt{n(n-1)}}.
\end{align*}
\end{frame}

\begin{frame}{Interpretação de um intervalo de confiança}
 \textbf{ATENÇÃO:} a interpretação de um intervalo é crucial.
 Muita gente confunde o que um intervalo de confiança significa!
 \begin{obs}[Um intervalo de confiança não é uma afirmação sobre o(s) parâmetro(s)!]
 \label{rmk:confidence_intervals_not_about_parameters}
  A afirmação probabilística da forma $\pr\left\{A(\bX) < g(\theta) <  B(\bX)\right\} = \gamma$ diz respeito à distribuição conjunta das variáveis aleatórias  $A(\bX)$ e $B(\bX)$ para um valor fixo de $\theta$ -- e, portanto, de $g(\theta)$.
 \end{obs}
 
 \begin{ideia}[Intervalos de confiança são procedimentos]
 \label{idea:confidence_intervals_are_procedures} 
 Como de costume na teoria ortodoxa (frequentista), o foco da construção de um intervalo confiança está em dar garantias probabilísticas~\textbf{com relação à \underline{distribuição dos dados}}.
 Dizer que $\pr\left\{A(\bX) < g(\theta) <  B(\bX)\right\} = \gamma$ é dizer que, se eu gerasse $M$ grande amostras aleatórias $\bX^{(1)}, \bX^{(2)}, \ldots, \bX^{(M)}$ de tamanho $n$ e construisse $M$ intervalos $I(\bX^{(1)}), I(\bX^{(2)}), \ldots, I(\bX^{(M)})$, eu esperaria encontrar:
 \begin{equation*}
  \frac{1}{M}\sum_{i=1}^M \mathbb{I}\left(g(\theta) \in I(\bX^{(i)}) \right) \approx \gamma.
 \end{equation*}
\end{ideia}
\end{frame}

\begin{frame}{I see $g(\theta)$...}

\begin{figure}[!ht]
\label{fig:freq_meme}
\begin{center}
\includegraphics[scale=0.45]{figures/freq_meme.jpg} 
\end{center} 
\end{figure} 
\end{frame}

\begin{frame}{Intervalos unilaterais}

Em várias situações, estamos interessados em uma cota superior ou inferior para $g(\theta)$.
\begin{defn}[Intervalo de confiança unilateral]
\label{def:oneSided_CIs}
  Seja $\bX = \{ \rs \}$ uma amostra aleatória, cada variável aleatória com p.d.f. $f(x\mid \theta)$, e considere uma função real $g(\theta)$.
  Seja $A(\bX)$ uma estatística que, para todo $\theta \in \Omega$, valha
  \begin{equation*}
   \pr\left\{A(\bX) < g(\theta)\right\} \geq \gamma,
  \end{equation*}
dizemos que o intervalo aleatório $(A(\bX), \infty)$ é chamado um intervalo de confiança~\textbf{unilateral} de $100\gamma\%$ para $g(\theta)$, ou, ainda, um intervalo de confiança~\textbf{inferior} de $100\gamma\%$ para $g(\theta)$.
O intervalo $(-\infty, B(\bX))$, com
  \begin{equation*}
   \pr\left\{g(\theta) < B(\bX) \right\} \geq \gamma,
  \end{equation*}
é definido de forma análoga, e é chamado de intervalo de confiança~\textbf{superior} de $100\gamma\%$ para $g(\theta)$.
Se a desigualdade é uma igualdade para todo $\theta \in \Omega$, os intervalos são chamados~\underline{exatos}.
\end{defn}
\end{frame}

\begin{frame}{Quantidade pivotal}
 O conceito de quantidade pivotal é útil na construção de intervalos de confiança.
 \begin{defn}[Quantidade pivotal]
 \label{def:pivotal_quantity}
   Seja $\bX = \{ \rs \}$ uma amostra aleatória com p.d.f. $f(x\mid \theta)$.
   Seja $V(\bX, \theta)$ uma variável aleatória cuja distribuição é~\textbf{a mesma} para todo $\theta \in \Omega$.
   Dizemos que $V(\bX, \theta)$ é uma~\textbf{quantidade pivotal}.
 \end{defn}
Podemos utilizar quantidades pivotais para construir intervalos de confiança.
Considere uma função $r(v, \bx)$ tal que
\[ r(V(\bX, \theta), \bX) = g(\theta). \]
\end{frame}

\begin{frame}{Construindo ICs a partir de quantidades pivotais}
 Vamos ver como usar $r(v, \bx)$ para construir um intervalo de confiança.
 \begin{theo}[Intervalos de confiança a partir de uma quantidade pivotal]
 \label{thm:CIs_from_pivotal}
     Seja $\bX = \{ \rs \}$ uma amostra aleatória com p.d.f. $f(x\mid \theta)$.
     Suponha que existe uma quantidade pivotal $V$, com c.d.f.~\underline{contínua} $G$.
     Assuma que existe $r(v, \bx)$, estritamente crescente em $v$ para todo $\bx$.
     Finalmente, tome $ 0 < \gamma < 1$ e $\gamma_1 < \gamma_2$ de modo que $\gamma_2 - \gamma_1 = \gamma$.
     Então as estatísticas
     \begin{align*}
      A(\bX) = r\left(G^{-1}(\gamma_1), \bX \right),\\
      B(\bX) = r\left(G^{-1}(\gamma_2), \bX \right),
     \end{align*}
são os limites de um intervalo de confiança de $100\gamma\%$ para $g(\theta)$.
 \end{theo}
\textbf{Prova:} Usar a monotonicidade de $r$ e de $G$ e notar que 
$$\pr(A(\bX) = g(\theta)) = \pr\left(V(\bX, \theta) = G^{-1}(\gamma_1)\right) = 0,$$
e que o mesmo vale para $B(\bX)$.
Ver Teorema 8.5.3 em DeGroot.
\end{frame}

\begin{frame}{Exemplos}
 \begin{itemize}
  \item Exponencial;
  \item Normal, $\mu$ e $\sigma^2$ desconhecidas;
  \item Normal, $\sigma^2$ conhecida;
 \end{itemize}
\end{frame}

\begin{frame}{Exemplos}
 \begin{itemize}
  \item Exponencial: $\theta S \sim \operatorname{Gama}(n, 1)$;
  \item Normal, $\mu$ e $\sigma^2$ desconhecidas:
\begin{itemize}
 \item $ \frac{\Sm - \mu}{\hat{\sigma}^\prime} \sim \operatorname{Student}(n-1)$;
 \item $\frac{\sum_{i=1}^n \left(X_i - \Sm \right)^2}{\sigma^2}\sim \operatorname{Qui-quadrado}(n-1)$; 
\end{itemize}
  \item Normal, $\sigma^2$ conhecida: $\frac{\Sm - \mu}{\sqrt{\frac{\sigma^2}{n}}} \sim \operatorname{Normal}(0, 1)$;
 \end{itemize}
\end{frame}

\begin{frame}{Limitações}
Intervalos de confiança estão entre as ferramentas mais importantes da Estatística.
Isso não quer dizer que não tenham limitações importantes.
\begin{itemize}
 \item Interpretação;
 Uma vez observados $a(\bx)$ e $b(\bx)$, não é correto dizer que $g(\theta)$ mora em $(a, b)$ com probabilidade $\gamma$.  
 ``Antes de observarmos o valor tomado por $\bX$, há probabilidade $\gamma$ de que o intervalo $I(\bX)$, construído a partir da amostra $\bX$, inclui $g(\theta)$.''
 
 Em geral falamos de~\textbf{confiança} $\gamma$ do intervalo $I(\bX)$.
 
 \item Uso da informação;
Uma vez que observamos $I(\bx) = (a(\bx), b(\bx))$, pode haver informação extra sobre se $I(\bx)$ cobre $g(\theta)$ ou não, mas não existe maneira canônica de ajustar o nível de confiança $\gamma$ à luz desta nova informação.
Ver exemplo 8.5.11 em DeGroot. 
\end{itemize}

\end{frame}


\begin{frame}{O que aprendemos?}
\begin{itemize}

  \item[\faLightbulbO] Intervalos de confiança;    
  
   ``Um intervalo $(A(\bX), B(\bX))$  de confiança de $100\gamma\%$ para $g(\theta)$ é tal que $\pr\left[ A(\bX) < g(\theta) <  B(\bX) \right] \geq \gamma$'';
   
  \item[\faLightbulbO] Um intervalo de confiança é uma afirmação probabilística sobre~\textbf{as estatísticas} $A(\bX)$ e $B(\bX)$ a partir da~\textbf{distribuição conjunta dos dados};
  
  \item[\faLightbulbO] Quantidade pivotal
  ``Uma quantidade pivotal é uma função $V(\bX, \theta)$ cuja distribuição não depende de $\theta$''
     
   \item[\faLightbulbO] Intervalos de confiança podem ser construídos a partir de quantidades pivotais;  
  
  \end{itemize}
 \end{frame}

\begin{frame}{Leitura recomendada}
\begin{itemize}
 \item[\faBook] DeGroot seção 8.5;
 \item[\faBook] $^\ast$ Casella \& Berger (2002), seção 9.2.
%  \item[\faBook] $^\ast$ Schervish (1995), capítulo 7.
 \item[\faForward] Próxima aula: DeGroot, seção 9.1;
 \item {\large\textbf{Exercícios recomendados}}
 \begin{itemize}
  \item[\faBookmark] DeGroot.
  \begin{itemize}
   \item Seção 8.5: 1, 4, 5 e 6.
  \end{itemize}   
  \end{itemize}
 \end{itemize} 
\end{frame}
