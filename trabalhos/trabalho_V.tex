\documentclass[a4paper,10pt, notitlepage]{report}
\usepackage[utf8]{inputenc}
\usepackage{natbib}
\usepackage{amssymb}
\usepackage{amsmath}
\usepackage{enumitem}
\usepackage[portuguese]{babel}

\newcommand{\indep}{\perp \!\!\! \perp} %% indepence
\newcommand{\pr}{\operatorname{Pr}} %% probability
\newcommand{\vr}{\operatorname{Var}} %% variance
\newcommand{\rs}{X_1, X_2, \ldots, X_n} %%  random sample
\newcommand{\irs}{X_1, X_2, \ldots} %% infinite random sample
\newcommand{\rsd}{x_1, x_2, \ldots, x_n} %%  random sample, realised
\newcommand{\Sm}{\bar{X}_n} %%  sample mean, random variable
\newcommand{\sm}{\bar{x}_n} %%  sample mean, realised
\newcommand{\Sv}{\bar{S}^2_n} %%  sample variance, random variable
\newcommand{\sv}{\bar{s}^2_n} %%  sample variance, realised
\newcommand{\bX}{\boldsymbol{X}} %%  random sample, contracted form (bold)
\newcommand{\bx}{\boldsymbol{x}} %%  random sample, realised, contracted form (bold)
\newcommand{\bT}{\boldsymbol{T}} %%  Statistic, vector form (bold)
\newcommand{\bt}{\boldsymbol{t}} %%  Statistic, realised, vector form (bold)
\newcommand{\emv}{\hat{\theta}_{\text{EMV}}}

% Title Page
\title{Trabalho V (extra): Regressão à média.}
\author{Disciplina: Inferência Estatística \\ Professor: Luiz Max de Carvalho}

\begin{document}
\maketitle

\textbf{Data de Entrega: 30/11/2021 às 23:59h.}

\section*{Orientações}
\begin{itemize}
 \item Enuncie e prove (ou indique onde se pode encontrar a demonstração) de~\underline{todos} os resultados não triviais necessários aos argumentos apresentados;
 \item Lembre-se de adicionar corretamente as referências bibliográficas que utilizar e referenciá-las no texto;
 \item Equações e outras expressões matemáticas também recebem pontuação;
 \item Você pode utilizar figuras, tabelas e diagramas para melhor ilustrar suas respostas;
 \item Indique com precisão os números de versão para quaisquer software ou linguagem de programação que venha a utilizar para responder às questões\footnote{Não precisa detalhar o que foi usado para preparar o documento com a respostas. Recomendo a utilização do ambiente LaTeX, mas fique à vontade para utilizar outras ferramentas.};
 \end{itemize}


\section*{Introdução}

O fenômeno de regressão à média é um daqueles ``paradoxos'' probabilísticos que suscita discussão mesmo entre especialistas até hoje.
Seja na querela sobre futebol no bar ou sobre as tendências macroeconômicas de momento, a regressão à média está presente no imaginário de todos aqueles que buscam apoiar seus argumentos em dados.

Uma ótima dica de leitura sobre a regressão à média e outras maneiras de como evitar  ser enganado pela aleatoriedade é o livro ``O andar do bêbado: Como o acaso determina nossas vidas'', do físico estadunidense Leonard Mlodinow (1954-).

\section*{Questões}
\begin{enumerate}
 \item Explique o que é regressão à média; utilize ``matematiquês'' se achar conveniente;
 \item Prove que o fenômeno acontece mesmo quando as variáveis em questão têm a mesma variância.
 Resolva o exercício 19 da seção 11.2 de DeGroot;
 \item Comente sobre a relevância do fenômeno no cotidiano.
\end{enumerate}

 
% \bibliographystyle{apalike}
% \bibliography{refs}

\end{document}          
