\documentclass[a4paper,10pt, notitlepage]{report}
\usepackage[utf8]{inputenc}
\usepackage{natbib}
\usepackage{amssymb}
\usepackage{amsmath}
\usepackage{enumitem}
\usepackage{xcolor}
\usepackage{cancel}
\usepackage{mathtools}
\usepackage[portuguese]{babel}

\newcommand{\indep}{\perp \!\!\! \perp} %% indepence
\newcommand{\pr}{\operatorname{Pr}} %% probability
\newcommand{\vr}{\operatorname{Var}} %% variance
\newcommand{\rs}{X_1, X_2, \ldots, X_n} %%  random sample
\newcommand{\irs}{X_1, X_2, \ldots} %% infinite random sample
\newcommand{\rsd}{x_1, x_2, \ldots, x_n} %%  random sample, realised
\newcommand{\Sm}{\bar{X}_n} %%  sample mean, random variable
\newcommand{\sm}{\bar{x}_n} %%  sample mean, realised
\newcommand{\Sv}{\bar{S}^2_n} %%  sample variance, random variable
\newcommand{\sv}{\bar{s}^2_n} %%  sample variance, realised
\newcommand{\bX}{\boldsymbol{X}} %%  random sample, contracted form (bold)
\newcommand{\bx}{\boldsymbol{x}} %%  random sample, realised, contracted form (bold)
\newcommand{\bT}{\boldsymbol{T}} %%  Statistic, vector form (bold)
\newcommand{\bt}{\boldsymbol{t}} %%  Statistic, realised, vector form (bold)
\newcommand{\emv}{\hat{\theta}_{\text{EMV}}}
\DeclarePairedDelimiter\ceil{\lceil}{\rceil}
\DeclarePairedDelimiter\floor{\lfloor}{\rfloor}
\newcommand{\rpl}{\mathbb{R}_+}
% Title Page
\title{Trabalho V: Desenho amostral para controlar as probabilidades de erro de testes de hipótese.}
\author{Disciplina: Inferência Estatística \\ Professor: Luiz Max de Carvalho}

\begin{document}
\maketitle

\textbf{Data de Entrega: 30 de Novembro de 2022.}

\section*{Orientações}
\begin{itemize}
 \item Enuncie e prove (ou indique onde se pode encontrar a demonstração) de~\underline{todos} os resultados não triviais necessários aos argumentos apresentados;
 \item Lembre-se de adicionar corretamente as referências bibliográficas que utilizar e referenciá-las no texto;
 \item Equações e outras expressões matemáticas também recebem pontuação;
 \item Você pode utilizar figuras, tabelas e diagramas para melhor ilustrar suas respostas;
 \item Indique com precisão os números de versão para quaisquer software ou linguagem de programação que venha a utilizar para responder às questões\footnote{Não precisa detalhar o que foi usado para preparar o documento com a respostas. Recomendo a utilização do ambiente LaTeX, mas fique à vontade para utilizar outras ferramentas.};
 \end{itemize}


\paragraph{Notação:} Como convenção adotamos $\mathbb{R} = (-\infty, \infty)$, $\rpl = (0, \infty)$ e $\mathbb{N} = \{1, 2, \ldots \}$.

\paragraph{Motivação: Entre os vários fatores a serem considerados na construção de um teste estatístico, a capacidade de detectar um efeito caso ele esteja presente é um das mais importantes.
Em algumas situações é possível determinar o tamanho de amostra necessário para controlar as probabilidades de erro do teste em questão.
E é exatamente isso que faremos neste exercício.
}

\section*{É pra medir \textit{quantos} mesmo, chefe ?!}

Suponha que os seus dados vêm de uma distribuição normal com parâmetros $\mu$ e $\sigma^2$.
Você tem acesso à média amostral, $\bar{X}_n = n^{-1} \sum_{i=1}^n X_i$ e à variância amostral, $S_2 = (n-1)^{-1} \sum_{i=1}^n (X_i - \bar{X}_n)^2$.
Você recebeu a tarefa de desenhar um teste estatístico para testar as hipóteses
\begin{align*}
    H_0 &:  \mu = \mu_0,\\
    H_1 &: \mu \neq \mu_0.
\end{align*}
\begin{enumerate}
    \item Suponha que $\sigma^2$ é \underline{conhecida} e considere o teste
    \begin{equation*}
        \delta_c = \begin{cases}
            \textrm{Rejeitar} \: H_0 \: \textrm{quando} \: |\bar{X}_n - \mu_0|/\sigma \geq c,\\
            \textrm{Falhar em rejeitar} \: H_0 \: \textrm{caso contrário}.
        \end{cases}
    \end{equation*}
    Determine o valor de $c$ para que o tamanho de $\delta_c$ seja $\alpha = 0.01$.
    \item Vamos agora supor $\sigma^2$ \underline{desconhecida}.
    Defina $\hat{\sigma}^\prime = \sqrt{S_2}$ e considere o teste     \begin{equation*}
        \delta_k^\prime = \begin{cases}
            \textrm{Rejeitar} \: H_0 \: \textrm{quando} \: |\sqrt{n}(\bar{X}_n - \mu_0)/\hat{\sigma}^\prime|  \geq k,\\
            \textrm{Falhar em rejeitar} \: H_0 \: \textrm{caso contrário}.
        \end{cases}
       \end{equation*}
         Determine o valor de $k$ para que o tamanho de $\delta_k^\prime$ seja $\alpha = 0.01$.
    \item Para cada um dos testes acima ($\delta_c$ e $\delta^\prime_k$), determine o tamanho amostral ($n$) tal que o teste tenha poder de $0.95$ em $\mu + \sigma$, isto é $\pi(\mu + \sigma | \delta_c) = 0.95$ e $\pi(\mu + \sigma | \delta_k) = 0.95$.
    Compare os tamanhos amostrais necessários e discuta se são diferentes e por quê.
\end{enumerate}

% \bibliographystyle{apalike}
% \bibliography{refs}

\end{document}  
