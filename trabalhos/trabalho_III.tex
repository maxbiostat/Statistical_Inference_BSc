\documentclass[a4paper,10pt, notitlepage]{report}
\usepackage[utf8]{inputenc}
\usepackage{natbib}
\usepackage{amssymb}
\usepackage{amsmath}
\usepackage{enumitem}
\usepackage[portuguese]{babel}


% Title Page
\title{Trabalho III: o comportamento assintótico de estimadores eficientes.}
\author{Disciplina: Inferência Estatística \\ Professor: Luiz Max de Carvalho}

\begin{document}
\maketitle

\textbf{Data de Entrega: 20 de Outubro de 2021.}

\section*{Orientações}
\begin{itemize}
 \item Enuncie e prove (ou indique onde se pode encontrar a demonstração) de~\underline{todos} os resultados não triviais necessários aos argumentos apresentados;
 \item Lembre-se de adicionar corretamente as referências bibliográficas que utilizar e referenciá-las no texto;
 \item Equações e outras expressões matemáticas também recebem pontuação;
 \item Você pode utilizar figuras, tabelas e diagramas para melhor ilustrar suas respostas;
 \item Indique com precisão os números de versão para quaisquer software ou linguagem de programação que venha a utilizar para responder às questões\footnote{Não precisa detalhar o que foi usado para preparar o documento com a respostas. Recomendo a utilização do ambiente LaTeX, mas fique à vontade para utilizar outras ferramentas.};
 \item Este trabalho é \underline{longo}.
 Sugiro fortemente começar a fazer assim que possível.
 \end{itemize}


\section*{Introdução}

Como aprendemos até agora, existem vários critérios de otimalidade para a construção e avaliação de estimadores.
Um conceito fundamental é o de variância mínima, ou eficiência, uma propriedade de estimadores não-viesados.

Na vida real, no entanto, mesmo um estimador viesado ou ineficiente pode ser útil.
Um dos aspectos que buscamos estudar é o comportamento~\textit{assintótico} de estimadores, isto é, o que acontece quando o tamanho de amostra, $n$, tende ao infinito.
No que se segue, vamos estudar alguns resultados interessantes sobre o comportamento assintótico de estimadores e, utilizando simulações, investigar o seu comportamento empírico.

\section*{Questões}

\subsection*{Parte I: lidando com estimadores eficientes}
\begin{enumerate}
 \item Considere um modelo estatístico paramétrico $f(x \mid \theta)$ com $f$ duas vezes diferenciável com respeito a $\theta$ e suporte independente de $\theta$.
 Seja $\boldsymbol{X}$ uma amostra aleatória de tamanho $n$ e $\delta(\boldsymbol{X})$ um estimador~\textbf{eficiente} de $g(\theta)$.
 Defina $E[\delta(\boldsymbol{X})] = m(\theta)$ de modo que $m^\prime(\theta) := \frac{d}{d\theta}m(\theta) \neq 0$ para todo $\theta \in \Omega$.
 Mostre que a distribuição de 
 $$
 \frac{\sqrt{nI(\theta)}}{m^\prime(\theta)}\left[\delta(\boldsymbol{X})-m(\theta)\right],
 $$
 é normal padrão, onde $I(\theta)$ é a informação de Fisher.
 
 \textit{Dica}: Lembre-se do método Delta.
 \item Seja $\boldsymbol{X}$ uma amostra aleatória de tamanho $n$ de uma distribuição Poisson com taxa $\mu$. 
 Mostre que o EMV para $\mu$ é eficiente.
 \item Tomando $\mu_0 = 0.5$, simule $100.000$ conjuntos de dados de $n=10$ observações com distribuição $\operatorname{Poisson}(\mu_0)$.
 Para cada simulação $\boldsymbol{X}^{(m)}$, $m = 1, 2,\ldots, 10^{5}$, compute o EMV, $\hat{\mu}^{(m)}$.
 Agora, compute a fração de simulações para as quais $\hat{\mu}^{(m)} \leq 0.55$.
 Esta é uma estimativa da função de distribuição empírica\footnote{ECDF, na sigla em inglês.} de $\hat{\mu}$, $\hat{F}(0.55)$.
 Compare $\hat{F}(0.55)$ com a aproximação assintótica derivada no item 1.
 Repita o experimento para $n=30$ e $n=100$.
 Para ajudar, aqui vai uma tabela a ser preenchida (utilize 3 dígitos de significância):
\begin{table}[!ht]
\begin{center}
 \begin{tabular}{ccc}
\hline
Tamanho de amostra ($n$) & CDF empírica & Aproximação normal \\ \hline
10 & 0.xxx & 0.xxx \\
30 & 0.xxx & 0.xxx \\
100 & 0.xxx & 0.xxx
\end{tabular}
\end{center}
\end{table}
%  \item \textit{Dica}: Vale lembrar que se $g$ é inversível e diferenciável, então
%  $$I(g(\theta)) = I(\theta)\left|\frac{d}{d\theta}g^{-1}(\theta)\right|.$$
\item Discuta o quão boa a aproximação normal é, e se a qualidade melhora à medida que $n$ cresce.
\end{enumerate}

\subsection*{Parte II: condições menos que ideais}

Agora vamos lidar com uma situação onde o estimador em questão não é eficiente.
O EMV, por exemplo, nem sempre é eficiente, mas podemos enunciar um resultado parecido com o da seção anterior. 
Sob condições de regularidade, temos que
 $$
 \sqrt{nI(\theta)}\left[\delta_{\text{EMV}}-\theta\right],
 $$
tem distribuição normal padrão,  isto é, que o EMV é~\textit{assintoticamente} eficiente.

\begin{enumerate}\addtocounter{enumi}{4}
 \item Tome $\boldsymbol{X}$ uma amostra aleatória de tamanho $n$ de uma distribuição exponencial com taxa $\theta$.
 Mostre que o EMV para $\theta$ é viesado e ineficiente.
 \item Mostre que
 $$
 \delta_{\text{EMV}}(\boldsymbol{X}) \sim \operatorname{Gama-inversa}(n, n\theta).
 $$
 \item Nesta situação, portanto, sabemos a função de distribuição do EMV exatamente.
 Vamos compará-la com a sua aproximação normal.
 Tomando $\theta = 2$, e $\delta^\ast = 3$, considere $\operatorname{Pr}(\delta(\boldsymbol{X}) \leq \delta^\ast)$ e preencha a tabela a seguir:
 \begin{table}[!ht]
\begin{center}
 \begin{tabular}{ccc}
\hline
Tamanho de amostra ($n$) & CDF exata & Aproximação normal \\ \hline
10 & 0.xxx & 0.xxx \\
30 & 0.xxx & 0.xxx \\
100 & 0.xxx & 0.xxx
\end{tabular}
\end{center}
\end{table}

 \textit{Dica}: Se não quiser utilizar pacotes especializados para computar a CDF exata, não precisa. 
 Basta lembrar que se $X \sim \operatorname{Gama}(\alpha, \beta)$, então $Y = 1/X$ tem distribuição $\operatorname{Gama-inversa}(\alpha, \beta)$, de modo que você consegue deduzir $\operatorname{Pr}(Y \leq y)$ a partir da função de distribuição de $X$, que está disponível em quase todos os pacotes estatísticos modernos. 
\end{enumerate}

\bibliographystyle{apalike}
\bibliography{refs}

\end{document}          
